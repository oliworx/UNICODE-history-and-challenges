% Literaturliste soll im Inhaltsverzeichnis auftauchen
\newpage
\addcontentsline{toc}{section}{Literatur}
  \begin{thebibliography}{}

%  Monographien:
%1.  Familienname des Verfassers
%2.  Vorname des Verfassers, vorzugsweise abgekürzt
%3.  Erscheinungsjahr in Klammern
%4.  vollständiger Titel des Werks
%5.  Auflage (Aufl.), wenn es sich nicht um die erste Auflage handelt
%6.  Erscheinungsort, der grundsätzlich dem Verlagsort entspricht (mehr als 3 Orte sind nicht zu nennen; stattdessen: u.a.) und / oder Verlag
%7.  Erscheinungsjahr
%8.  Handelt es sich bei der Literaturquelle um eine Dissertation, ist vor 5. die Abkürzung „Diss.“ Einzufügen

% Beispiele:
% Scholz, C. (2000): Personalmanagement, 5. Aufl., München 2000
% Lüdenbach, N., Hoffmann, W.-D. (2003): Haufe IAS-Kommentar, Freiburg 2003
% Ruhnke, K. (1995): Konzernbuchführung, Diss., Düsseldorf 1995

	\bibitem[Becker, J.D. (1988)]{Becker88} {\sl Unicode 88} , Xerox Corporation, Palo Alto, CA, 1988
	
	\textbf{Internetquellen:}
	 
	\bibitem[Andrew West (2007)]{West2007} {\sl A Brief History of CJK-C} URL: \url{http://babelstone.blogspot.de/2007/06/brief-history-of-cjk-c.html}, Abruf am 31.1.2015
	\bibitem[bayern-online.com (2015)]{BO} {\sl Baudot-Mehrfachtelegraf} URL: \url{http://www.bayern-online.com/v2261/artikel.cfm/203/Baudot-Mehrfachtelegraf.html}, Abruf am 29.1.2015
	\bibitem[IETF (1992)]{IETF} {\sl RFC 1345: Character Mnemonics and Character Sets} URL: \url{https://tools.ietf.org/html/rfc1345}, Abruf am 29.1.2015
	\bibitem[IETF (2003)]{IETF2003} {\sl RFC 3629: UTF-8, a transformation format of ISO 10646} URL: \url{http://tools.ietf.org/html/rfc3629}, Abruf am 30.1.2015
	\bibitem[IETF (2010)]{IETF2010} {\sl RFC 5891: Internationalized Domain Names in Applications (IDNA): Protocol} URL: \url{https://tools.ietf.org/html/rfc5891}, Abruf am 31.1.2015
	\bibitem[ITU (1988)]{ITU} {\sl INTERNATIONAL TELEGRAPH ALPHABET No. 2} URL: \url{https://www.itu.int/rec/dologin_pub.asp?lang=e&id=T-REC-S.1-198811-S!!PDF-E&type=items}, Abruf am 29.1.2015
	\bibitem[Pike, R.C. (2003)]{Pike2003} {\sl UTF-8 history} URL: \url{http://www.cl.cam.ac.uk/~mgk25/ucs/utf-8-history.txt}, Abruf am 29.1.2015
	\bibitem[Unicode Consortium (2009)]{UCC} {\sl Early Years of Unicode} URL: \url{http://unicode.org/history/earlyyears.html}, Abruf am 29.1.2015
	\bibitem[Unicode Consortium (2014a)]{UC72} {\sl The Unicode Standard Version 7.0 – Core Specification, Chapter 2: General Structure} URL: \url{http://www.unicode.org/versions/Unicode7.0.0/ch02.pdf}, Abruf am 30.1.2015
	\bibitem[Unicode Consortium (2014b)]{UC2014B} {\sl The Unicode Standard Version 7.0 – Tags} URL: \url{http://www.unicode.org/charts/PDF/UE0000.pdf}, Abruf am 31.1.2015
	\bibitem[Unicode Consortium (2014c)]{UC2014C} {\sl The Unicode Standard Version 7.0 – Core Specification, Appendix C: Relationship to ISO/IEC 10646} URL: \url{http://www.unicode.org/versions/Unicode7.0.0/appC.pdf}, Abruf am 29.1.2015
	\bibitem[Unicode Consortium (2014e)]{UC2014E} {\sl The Unicode Standard Version 7.0 – Core Specification, Appendix E: Han Unification History} URL: \url{http://www.unicode.org/versions/Unicode7.0.0/appE.pdf}, Abruf am 29.1.2015
	\bibitem[Unicode Consortium (2014f)]{UC2014F} {\sl Unicode Character Encoding Stability Policy} URL: \url{http://www.unicode.org/policies/stability_policy.html#Normalization}, Abruf am 31.1.2015
	\bibitem[Unicode Consortium (2014g)]{UC2014G} {\sl Unicode Technical Standard \#39 - UNICODE SECURITY MECHANISMS} URL: \url{http://www.unicode.org/reports/tr39/}, Abruf am 31.1.2015
	\bibitem[Unicode Consortium (2014h)]{UC2014H} {\sl Draft Emoji Annotations} URL: \url{http://www.unicode.org/Public/emoji/1.0/emoji-annotations.html}, Abruf am 31.1.2015
	\bibitem[Unicode Consortium (2014i)]{UC2014I} {\sl Frequently Asked Questions - Chinese and Japanese} URL: \url{	http://unicode.org/faq/han_cjk.html}, Abruf am 31.1.2015
	\bibitem[W3C (2011)]{W3C2011} {\sl Zeichencodierungen: grundlegende Konzepte} URL: \url{http://www.w3.org/International/articles/definitions-characters/}, Abruf am 31.1.2015
	\bibitem[W3Techs (2015)]{W3Tech} {\sl Historical yearly trends in the usage of character encodings for websites} URL: \url{http://w3techs.com/technologies/history_overview/character_encoding/ms/y}, Abruf am 31.1.2015
	\bibitem[Wikimedia Commons (2014)]{wcommons} {\sl Comparison of the characters {\CN 判, 逸, 骨}  (traditional variants) for Chinese (Mainland), Chinese (Taiwan), Japanese, Korean.} URL: \url{https://commons.wikimedia.org/wiki/File:Vergleich_zh-Hant-CN_zh-Hant-TW_ja-Hani_ko-Hani.svg}, Abruf am 31.1.2015

  \end{thebibliography}
