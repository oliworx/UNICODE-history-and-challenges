%Verzeichnisse

% vor dem Hauptteil römische Seitenzahlen verwenden
\pagenumbering{Roman}
\newpage
% dies ist die erste Seite bezüglich der Seitenzahlen
\setcounter{page}{1}

% Inhaltsverzeichnis
\tableofcontents
\newpage

% Abkürzungsverzeichnis
\phantomsection \addcontentsline{toc}{section}{Abkürzungsverzeichnis}
\section*{Abkürzungsverzeichnis}
\begin{acronym}[Xerox PARC] % für die Ausrichtung die längste Abk. hier in eckigen Klammern
 \acro{ASA}{American Standards Association, Vorläuferorganisation von ANSI}
 \acro{ANSI}{American National Standards Institute}
 \acro{Big5}{Zeichenkodierung für traditionelle chinesische Schriftzeichen, entwickelt von den 5 größten Computerfirmen Taiwans}
 \acro{BOM}{Byte Order Mark}
 \acro{BMP}{Basic Multilingual Plane, Mehrsprachige Basis-Ebene, Ebene 0 der Unicode-Zeichen}
 \acro{CJK}{Chinesisch, Japanisch, und Koreanisch: Sprachen mit Zeichen chinesischen Ursprungs}
 \acro{CJKV}{Chinesisch, Japanisch, Koreanisch, und (historische) Vietnamesische Schrift}
 \acro{DBCS}{Double Byte Character Set - Klasse von Zeichensätzen mit Ein- und Zweibyte-Zeichen}
 \acro{ECMA}{European Computer Manufacturers Association, Verband europäischer Computerhersteller}
 \acro{GB2312}{Guojia Biaozhun, nationaler Standard in China, Zeichensatz für vereinfachte chinesische Schriftzeichen}
 \acro{GB18030}{Guojia Biaozhun, nationaler Standard in China, Nachfolger von GB2312}
 \acro{IDN}{Internationalisierter Domainname, Domainname mit Nicht-ASCII-Zeichen}
 \acro{IEC}{International Electrotechnical Commission, Internationale Elektrotechnische Kommission, Normungsorganisation für Elektrotechnik und Elektronik}
 \acro{ISO}{International Organization for Standardization, Internationale Organisation für Normung}
 \acro{PUA}{Private Use Area, privat nutzbarer Bereich}
 \acro{PUP}{Private Use Planes, Ebenen für private Nutzung, Ebenen 15 und 16 von Unicode}
 \acro{SIP}{Supplementary Ideographic Plane, Ergänzende ideographische Ebene, Ebene 2 und 3 der Unicode-Zeichen}
 \acro{SMP}{Supplementary Multilingual Plane, Ergänzende mehrsprachige Ebene, Ebene 1 der Unicode-Zeichen}
 \acro{SSP}{Supplementary Special-purpose Plane, Ergänzende Ebene für spezielle Verwendungen, Ebene 14 der Unicode-Zeichen}
 \acro{UCS}{Universal Character Set, in ISO/IEC 10646 spezifizierte Norm}
 \acro{UCS-2}{Universal Character Set, Kodierung in 2 Byte}
 \acro{UCS-4}{Universal Character Set, Kodierung in 4 Byte}
 \acro{UTF-16}{Unicode Transformation Format 16 bit}
 \acro{UTS}{Unicode Technical Standard, vom Unicode Consortium veröffentlichte Spezifikation}
 \acro{Xerox PARC}{Xerox Palo Alto Research Center, Forschungszentrum der Firma Xerox im kalifornischen Palo Alto}

\end{acronym}

% ggf. Abbildungsverzeichnis auf eine neue Seite
 \newpage 
 \addcontentsline{toc}{section}{Abbildungsverzeichnis}
 \listoffigures

% ggf. Tabellenverzeichnis
% \addcontentsline{toc}{section}{Tabellenverzeichnis}
% \listoftables

\newpage
\pagenumbering{arabic}
